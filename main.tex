\documentclass[journal,12pt,twocolumn]{IEEEtran}

\usepackage{setspace}
\usepackage{gensymb}

\singlespacing


\usepackage[cmex10]{amsmath}

\usepackage{amsthm}

\usepackage{mathrsfs}
\usepackage{txfonts}
\usepackage{stfloats}
\usepackage{bm}
\usepackage{cite}
\usepackage{cases}
\usepackage{subfig}

\usepackage{longtable}
\usepackage{multirow}

\usepackage{enumitem}
\usepackage{mathtools}
\usepackage{steinmetz}
\usepackage{tikz}
\usepackage{circuitikz}
\usepackage{verbatim}
\usepackage{tfrupee}
\usepackage[breaklinks=true]{hyperref}
\usepackage{graphicx}
\usepackage{tkz-euclide}
\usepackage{float}

\usetikzlibrary{calc,math}
\usepackage{listings}
\usepackage{color} %%
\usepackage{array} %%
\usepackage{longtable} %%
\usepackage{calc} %%
\usepackage{multirow} %%
\usepackage{hhline} %%
\usepackage{ifthen} %%
\usepackage{lscape}
\usepackage{multicol}
\usepackage{chngcntr}

\DeclareMathOperator*{\Res}{Res}

\renewcommand\thesection{\arabic{section}}
\renewcommand\thesubsection{\thesection.\arabic{subsection}}
\renewcommand\thesubsubsection{\thesubsection.\arabic{subsubsection}}

\renewcommand\thesectiondis{\arabic{section}}
\renewcommand\thesubsectiondis{\thesectiondis.\arabic{subsection}}
\renewcommand\thesubsubsectiondis{\thesubsectiondis.\arabic{subsubsection}}


\hyphenation{op-tical net-works semi-conduc-tor}
\def\inputGnumericTable{} %%

\lstset{
%language=C,
frame=single,
breaklines=true,
columns=fullflexible
}
\begin{document}


\newtheorem{theorem}{Theorem}[section]
\newtheorem{problem}{Problem}
\newtheorem{proposition}{Proposition}[section]
\newtheorem{lemma}{Lemma}[section]
\newtheorem{corollary}[theorem]{Corollary}
\newtheorem{example}{Example}[section]
\newtheorem{definition}[problem]{Definition}

\newcommand{\BEQA}{\begin{eqnarray}}
\newcommand{\EEQA}{\end{eqnarray}}
\newcommand{\define}{\stackrel{\triangle}{=}}
\bibliographystyle{IEEEtran}
\providecommand{\mbf}{\mathbf}
\providecommand{\pr}[1]{\ensuremath{\Pr\left(#1\right)}}
\providecommand{\qfunc}[1]{\ensuremath{Q\left(#1\right)}}
\providecommand{\sbrak}[1]{\ensuremath{{}\left[#1\right]}}
\providecommand{\lsbrak}[1]{\ensuremath{{}\left[#1\right.}}
\providecommand{\rsbrak}[1]{\ensuremath{{}\left.#1\right]}}
\providecommand{\brak}[1]{\ensuremath{\left(#1\right)}}
\providecommand{\lbrak}[1]{\ensuremath{\left(#1\right.}}
\providecommand{\rbrak}[1]{\ensuremath{\left.#1\right)}}
\providecommand{\cbrak}[1]{\ensuremath{\left\{#1\right\}}}
\providecommand{\lcbrak}[1]{\ensuremath{\left\{#1\right.}}
\providecommand{\rcbrak}[1]{\ensuremath{\left.#1\right\}}}
\theoremstyle{remark}
\newtheorem{rem}{Remark}
\newcommand{\sgn}{\mathop{\mathrm{sgn}}}
\providecommand{\abs}[1]{\left\vert#1\right\vert}
\providecommand{\res}[1]{\Res\displaylimits_{#1}}
\providecommand{\norm}[1]{\left\lVert#1\right\rVert}
%\providecommand{\norm}[1]{\lVert#1\rVert}
\providecommand{\mtx}[1]{\mathbf{#1}}
\providecommand{\mean}[1]{E\left[ #1 \right]}
\providecommand{\fourier}{\overset{\mathcal{F}}{ \rightleftharpoons}}
%\providecommand{\hilbert}{\overset{\mathcal{H}}{ \rightleftharpoons}}
\providecommand{\system}{\overset{\mathcal{H}}{ \longleftrightarrow}}
%\newcommand{\solution}[2]{\textbf{Solution:}{#1}}
\newcommand{\solution}{\noindent \textbf{Solution: }}
\newcommand{\cosec}{\,\text{cosec}\,}
\providecommand{\dec}[2]{\ensuremath{\overset{#1}{\underset{#2}{\gtrless}}}}
\newcommand{\myvec}[1]{\ensuremath{\begin{pmatrix}#1\end{pmatrix}}}
\newcommand{\mydet}[1]{\ensuremath{\begin{vmatrix}#1\end{vmatrix}}}
\numberwithin{equation}{subsection}
\makeatletter
\@addtoreset{figure}{problem}
\makeatother
\let\StandardTheFigure\thefigure
\let\vec\mathbf
\renewcommand{\thefigure}{\theproblem}
\def\putbox#1#2#3{\makebox[0in][l]{\makebox[#1][l]{}\raisebox{\baselineskip}[0in][0in]{\raisebox{#2}[0in][0in]{#3}}}}
\def\rightbox#1{\makebox[0in][r]{#1}}
\def\centbox#1{\makebox[0in]{#1}}
\def\topbox#1{\raisebox{-\baselineskip}[0in][0in]{#1}}
\def\midbox#1{\raisebox{-0.5\baselineskip}[0in][0in]{#1}}
\vspace{3cm}
\title{Assignment No.1}
\author{Sravani sandhya}
\maketitle
\newpage
\bigskip
\renewcommand{\thefigure}{\theenumi}
\renewcommand{\thetable}{\theenumi}
\begin{abstract}
This is a simple document to learn about writing vectors and matrices using latex, draw figures using Python, Latex.
\end{abstract}
%Download all python codes 
%
%\begin{lstlisting}
%svn co https://github.com/JayatiD93/trunk/My_solution_design/codes
%\end{lstlisting}
Download all and latex-tikz codes from 
%
\begin{lstlisting}
svn co https://github.com/sravani706/Assignment-1_new.git
\end{lstlisting}
%
\section{Vectors\\(CBSE-Math-X-2008. 30/2/2-Q.23)}
\renewcommand{\theequation}{\theenumi}
\begin{enumerate}[label=\thesection.\arabic*.,ref=\thesection.\theenumi]
\numberwithin{equation}{enumi}
\item Represent the following pair of equation graphically and write the coordinates of points where the line is intersect $y$ axis.
\begin{align}
    \vec{x+3y-6=0} \\
\vec{2x-3y-12=0} 
\end{align} 
\solution\begin{enumerate}
    \item Line x+3y-6=0  can be represented in vector form as,
    \begin{align}\label{eq:1}
        \myvec{1&3}\vec{x}=6
    \end{align}
    \item Line 2x-3y-12=0  can be represented in vector form as,
    \begin{align}\label{eq:2}
        \myvec{2&-3}\vec{x}=12
    \end{align}
\item Also the equation of y axis is
\begin{align}\label{eq:3}
        \myvec{1&0}\vec{x}=0
    \end{align}
Let line \eqref{eq:1} and line \eqref{eq:2} meet at point P.Then,
\begin{align}
    \myvec{1 & 3 \\ 2 & -3}\vec{P}=
\myvec{6 \\ 12}\\
\vec{P}=\myvec{1 & 3 \\ 2 & -3}^{-1}\myvec{6 \\ 12}\\
\vec{P}=\myvec{6\\0}
\end{align}

Let line \eqref{eq:1} and line \eqref{eq:3} meet at point Q.Then,
\begin{align}
    \myvec{1 & 3 \\ 1 & 0}\vec{Q}=
\myvec{6 \\ 0}\\
\vec{Q}=\myvec{1 & 3 \\ 1 & 0}^{-1}\myvec{6 \\ 0}\\
\vec{Q}=\myvec{0\\2}
\end{align}
Let line \eqref{eq:2} and line \eqref{eq:3} meet at point R.Then,
\begin{align}
    \myvec{2 & -3 \\ 1 & 0}\vec{R}=
\myvec{12 \\ 0}\\
\vec{R}=\myvec{2 & -3 \\ 1 & 0}^{-1}\myvec{12 \\ 0}\\
\vec{R}=\myvec{0\\-4}
\end{align}
So \triangle PQR is formed by intersection of \eqref{eq:1},\\
\eqref{eq:2}and \eqref{eq:3}

\numberwithin{figure}{section}
\begin{figure}[ht]
    \centering
    \includegraphics[width=\columnwidth]{line.pdf}
    \caption{Graphical solution}
    \label{Graphical solution}
\end{figure}

\end{document}
